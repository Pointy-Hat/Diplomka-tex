\documentclass[12pt]{article}
\usepackage{amsmath}
\usepackage{amsfonts}
\usepackage[utf8]{inputenc}
\begin{document}
V tejto práci sa snažíme vybudovať algebraické modely v počítači použitím strojového učenia, konkrétne neurónových sietí. Začneme množinou axiómov ktoré popisujú funkcie, konštanty a relácie a použijeme ich na trénovanie neurónových sietí ktoré ich aproximujú. Každý prvok je reprezentovaný reálnym vektorom, aby na nich neurónové siete mohli operovať. Taktiež skúmame a porovnávame rôzne reprezentácie. Táto práca sa zaoberá hlavne grupami. Trénujeme neurónové reprezentácie pre cyklické (najjednoduchšie) a symetrické (najkomplikovanejšie) grupy. Ďalšou časťou tejto práce sú experimenty s rozšírením týchto natrénovaných modelov pomocou "algebraických" prvkov, podobne ako klasické rozšírenia racionálnych čísel, napr. $\mathbb{Q}[\sqrt{2}]$. 
\end{document}