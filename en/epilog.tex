\chapter*{Conclusion}
\addcontentsline{toc}{chapter}{Conclusion}
We have seen some promising results with regards to using neural networks to simulate particular mathematical models by learning on propositions that are true/false in them. We have successfully learned neural representations of groups, namely the cyclic and symmetric groups. Another focus of the work was building extensions to those models, relying on the learned functions.\\

For every model built here we used the multiplication table to learn the composition operation. Even despite the fact that the whole table is not needed (we have had good results even with 10\% of the table missing), prior knowledge of the structure is still required. In order to truly follow the ideas of the model theory, we would need to drop the table altogether. How to do this is currently not known and would be a subject to further experimentation.\\

Another place to improve the method shown here is the grounding, specifically the usage of handpicked representations. This would ideally also be eliminated, since it is another essential part of the model that relies on prior knowledge of the structures. For the self-finding of the groundings we could use recurrent neural networks, which are widely used for feature extraction. Another method that could improve performance is mutable grounding, i.e. grounding that could change during the learning process to better reflect the structure learned. However, this might be quite nontrivial.\\

A very large part of this thesis is model extension. Despite initial optimism stemming from the successes of the finite cyclic groups, the results have been rather lackluster. We speculate that this is caused by the fact that we used the multiplication table rather than the associativity axiom. The associativity obviously holds in the original universe, since the multiplication table had been learned quite efficiently. One way to ensure associativity on the extension as well would be introduction of axioms such as $(h\cdot a)\cdot b = h\cdot (a\cdot b)$ where $h$ is the extension element. However, training for general associativity - i.e. associativity on the whole domain $\mathbb{R}^n$ might slow down the learning process severely.\\

We approached the problem from two different angles. One, simpler, the discrete approach where each function is learned separately and the joint approach where all are learned together, making it more universal. The challenges described above are common for both approaches. We have also seen the limitations of the universal approach, as it fails with simpler cyclic groups. However, when we increased the complexity of the underlying group this performance gap was diminished. Also, the universal approach was quicker by about 60\%.\\

Because the work shown here is very early, there was little focus on the end goal - building an oracle that would gauge the probability that a given sentence is true. This would be a boon to the automated theorem proving community. Current trend is to use machine learning on the sentences themselves, thus skipping the models altogether. This approach has considerable limitations.

Unfortunately, the model-building process is very slow (order of hours on a home computer), therefore building an array of models for every problem would take some time. Classical Automated theorem proving competitions run in relatively short times (minutes), rendering this method rather unwieldy for usage in the state it is in right now. There is however a feasible niche for this model-based oracle in recent large-theory competitions and benchmarks and in theory building, i.e. expanding a given theory without a set goal. Large-theory benchmarks such as CASC LTB and the MPTP Challenge provide a large global time limit (days) for solving many related
problems. Machine learning of useful models for predicting the
validity of lemmas and conjectures could be very useful there.

Another area where the neural models could be useful is axiom selection, like SRASS described in \cite{model_axiom_selection}. Proving a conjecture $C$ from a large knowledge base usually needs only a subset of the axioms. Since automated theorem provers perform better on smaller axiom sets, finding such a subset is crucial. SRASS attempts this by numbering the axioms in the knowledge base $\varphi_1,\varphi_2,\dots$ and then finding a model for $\{\neg C,\varphi_1,\dots \varphi_n\}$ for ever larger $n$. If no such model exists, $\varphi_1,\dots \varphi_n$ is the desired subset on which the theorem prover can run.\\

The framework for the joint approach that was used here can be found in the Appendix. We will continue development further, later version will be available for download on https://github.com/Pointy-Hat/Neural-modelling/ .